\documentclass{article}
\usepackage{arxiv}

\usepackage[utf8]{inputenc}
\usepackage[russian]{babel}
\usepackage[english]{babel}
\usepackage[T1]{fontenc}
\usepackage{url}
\usepackage{booktabs}
\usepackage{amsfonts}
\usepackage{nicefrac}
\usepackage{microtype}
\usepackage{bm}
\usepackage{lipsum}
\usepackage{graphicx}
\usepackage{natbib}
\usepackage{wrapfig}
\usepackage{doi}
\usepackage{mathtools}
\usepackage[dvipsnames]{xcolor}

\title{Выбор субоптимальной архитектуры модели в режиме многозадачного обучения с примененимем методов символьного программирования}

\author{
    Набиев Мухаммадшариф \\
    Кафедра интеллектуальных систем\\
    МФТИ\\
    \texttt{nabiev.mf@phystech.edu} \\
    \And
    Бахтеев Олег \\
    Кафедра интеллектуальных систем\\
    МФТИ\\
    \texttt{} \\
}
\date{}

\renewcommand{\shorttitle}{Выбор субоптимальной архитектуры модели в режиме многозадачного обучения с примененимем методов символьного программирования}

\hypersetup{
pdftitle={Inductive bias in model selection},
pdfsubject={q-bio.NC, q-bio.QM},
pdfauthor={Muhammadsharif, Oleg Bakhteev},
pdfkeywords={model selection, evolutionary algorithm},
}

\begin{document}
\maketitle

\begin{abstract}
Концепция индуктивного смещения является фундаментальным принципом в области машинного обучения, заключающим в себе основные допущения, лежащие в основе методологии, используемой конкретной моделью в ее прогнозных усилиях, выходящих за рамки явно наблюдаемых данных.
% Inductive bias играет ключевую роль для обобщения модели, так как разнородные данные имеют свои уникальные отличительные черты, которые отличают одни данные от других. В данной работе мы предлагаем автоматизированный подход к выявлению inductive bias в наборе данных, используя модифицированную версию AutoML-Zero. AutoML-Zero - эволюционный алгоритм, который автоматически проектирует модели для решения задачи на заданных данных. Мы предлагаем следующие модификации \textcolor{red}{need to write more}

\end{abstract}

\keywords{model selection \and evolutionary algorithm}

\section{Введение} 
Inductive bias является фундаментальной концепцией в машинном обучении, определяющей предположения, которые модель делает для обобщения данных за пределами наблюдаемых примеров.[\textcolor{red}{need to ref papers}] 


Сфера машинного обучения в последние годы претерпела значительные изменения, вызванные развитием сложных моделей и алгоритмов, предназначенных для решения самых разнообразных задач. Однако проектирование и оптимизация таких моделей вручную зачастую требуют значительных усилий и времени, что стимулирует рост автоматизированных систем машинного обучения (AutoML). Однако автоматический поиск модели в все еще требует человеческого вмешательства при инициализации пространства поиска или указании заранее созданных шаблонов. Недавняя статья\citep{automl-zero} описывает подход, который требует значительно меньшей настройки. Алгоритм собирает модель из математических операций для решения задач. Такой подход не накладывает никакие ограничения на структуру модели.

TODO

С помощью модели AutoML-Zero сделали autobert\citep{autobert} и automl4robotics\citep{automl4robots}
    

\section{Постановка задачи}

\section{Описание модели} 

\section{Вычислительный эксперимент}
    
    \subsection{Данные}
  
    \subsection{Эксперимент}

    \subsection{Результаты эксперимента}

\section{Заключение}

\bibliographystyle{plain}
\bibliography{ib_in_model_selection}

\end{document}